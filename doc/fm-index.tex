\title{FM-Index}
\documentclass[12pt]{article}

\usepackage{fullpage}
\usepackage{graphicx}
\usepackage[colorlinks=true,urlcolor=blue,linkcolor=blue]{hyperref}

\urlstyle{same}

\graphicspath{
  {./doc/}
  % bazel output dir options
  {./bazel-out/k8-dbg/bin/doc/}
  {./bazel-out/k8-fastbuild/bin/doc/}
  {./bazel-out/k8-opt/bin/doc/}
}

\begin{document}

\maketitle

% Abstract

\tableofcontents

%\clearpage 

\section{Overview}

An FM-Index is a is a full text index that allows finding and counting the
 occurrences of a given string in a corpus.
This is done via a Burrows-Wheeler and a Wavelet Tree along with a few auxiliary
 data structures.

\section{Components}

\subsection{Burrows-Wheeler Transform}

In short, performing a Burrows-Wheeler Transform\footnote{\href{https://en.wikipedia.org/wiki/Burrows-Wheeler_transform}{Wikipedia: Burrows-Wheeler Transform}}
(BWT) on a string consists of sorting all possible rotations of that string and
 extracting the last character from each in order.

This is a reversible operation because the first column of the sorted set can be
 generated by rotating the last column (the BWT) into the first position and sorting.
From there, the columns can again be rotated and sorted resulting in the second
 column.
This can be repeated until the entire string is reconstructed.

This process can be done in a more space efficient manner by noting that, if a
 stable sort is used, the reordering of rows is the same for every sort operation.
Thus sorting the BWT gives a index-to-index mapping that, when walked, generates
 the original string.

\subsubsection{Example:}
\textbf{Input:} {\tt hello-abbaca!}

\vspace{0.75em}
\noindent\begin{tabular}{ c|c|c }
  Rotated: & Sorted: \\ 
  \hline
  {\tt hello-abbaca!} & {\tt !hello-abbaca} & {\tt a} \\
  {\tt !hello-abbaca} & {\tt -abbaca!hello} & {\tt o} \\
  {\tt a!hello-abbac} & {\tt a!hello-abbac} & {\tt c} \\
  {\tt ca!hello-abba} & {\tt abbaca!hello-} & {\tt -} \\
  {\tt aca!hello-abb} & {\tt aca!hello-abb} & {\tt b} \\
  {\tt baca!hello-ab} & {\tt baca!hello-ab} & {\tt b} \\
  {\tt bbaca!hello-a} & {\tt bbaca!hello-a} & {\tt a} \\
  {\tt abbaca!hello-} & {\tt ca!hello-abba} & {\tt a} \\
  {\tt -abbaca!hello} & {\tt ello-abbaca!h} & {\tt h} \\
  {\tt o-abbaca!hell} & {\tt hello-abbaca!} & {\tt !} \\
  {\tt lo-abbaca!hel} & {\tt llo-abbaca!he} & {\tt e} \\
  {\tt llo-abbaca!he} & {\tt lo-abbaca!hel} & {\tt l} \\
  {\tt ello-abbaca!h} & {\tt o-abbaca!hell} & {\tt l} \\
\end{tabular}

\vspace{0.75em}
\noindent\textbf{Output:} {\tt aoc-bbaah!ell}

\subsection{Wavelet Tree}
A Wavelet Tree\footnote{\href{https://en.wikipedia.org/wiki/Wavelet_Tree}{Wikipedia: Wavelet Tree}}
 is a data structure that allows counting the occurrence of any given character
 in a given prefix of a string in $O(1)$ time and using $O(n)$ space.
This is accomplished by way of a binary tree indexed on the bits of the character.
Each node holds a bit vector for which child each position becomes part of and
 periodic cached counts.

This tree, is stored in an array encoded in the same way as heap mapping is
 traditionally done.
The left and right child of the node at position $i$ are at positions $2i + 1$
 and $2i + 2$.

To generate a count, the tree is walked for the bits of the characters in question.
At each node, the pop-count-before-position-$i_n$ is found starting with a cached
 count and adding the popcount of the remaining locations.
This sum then becomes the $i_{n+1}$ at the next node.

Note that this works regardless of what character is at position $i_0$.

\subsubsection{Example:}
\includegraphics[height=4cm]{wavelet}

\section{FM-Index}

The FM-Index is inspected by considering the BWT as a representation of a
 hypothetical matrix containing all rotations of the corpus.
To find a string (e.g. "abbaz"), the string is walked backwards keeping track of
 the span of rows in the "matrix" that start with the suffix so far inspected.
At the first step, every row starts with the empty string so the span is everything.
At each subsequent step, the count of rows, above and inside the span (from the
 the prior step) which has as their last character the next character in the
 query is found.
These counts, and the index of the first row starting with that character, give
 the span for the next step.
This can be iterated backwards to test for any given string.

Once the target string is processed, the resulting span of locations can be
 converted to locations in the original string by walking from each location,
 using the BTW to find the preceding character, until a known location is found
 (e.g. via a table mapping the indexes of newlines or null character in the BTW
 to the corresponding location in the original corpus).

\end{document}


